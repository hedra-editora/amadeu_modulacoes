\chapter*{}

\vspace*{\fill}

\thispagestyle{empty}

\epigraph{Os indivíduos tornam-se `dividuais', divisíveis, e as massas
tornaram-se amostras, dados, mercados ou `bancos'.}{(Deleuze, \emph{Post-scriptum sobre as sociedades de controle})}

\epigraph{Sublinhou-se recentemente a que ponto o exercício do poder moderno não
se reduzia à alternativa clássica ``repressão ou ideologia'', mas
implicava processos de normalização, de modulação, de modelização, de
informação, que se apóiam na linguagem, na percepção, no desejo, no
movimento, etc., e que passam por microagenciamentos.}{(Gilles Deleuze; Félix Guattari, \emph{Mil platôs -- capitalismo e esquizofrenia, vol. 5})}

\epigraph{A expressão e a efetuação dos mundos e das subjetividades neles
inseridas, a criação e realização do sensível (desejos, crenças,
inteligências) antecedem a produção econômica. A guerra econômica
travada em um nível planetário é assim uma guerra estética, sob vários
aspectos.}{(Lazzarato, \emph{As Revoluções do Capitalismo})}
