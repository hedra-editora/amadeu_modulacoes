\chapter{Introdução}

Este livro traz reflexões sobre o conceito de modulação. Retirado do
universo deleuzeano, a perspectiva da modulação está longe da maturidade
e de uma aplicação consensual. Foi em busca de uma definição mais
precisa, operacional e consistente que surgiu a ideia desta publicação.
As pesquisadoras e pesquisadores aqui reunidos tentam explorar as
possibilidades da modulação em suas pesquisas. Assim, apresentam pontos
de vista sobre o conceito em construção, nem sempre em uma mesma
direção, muitas vezes em sentidos opostos. Mas, a motivação é trazer o
debate, a dúvida e a polêmica, ao invés de mostrar apenas os pontos de
concordância.

Nas sociedades de controle, conectadas por tecnologias cibernéticas,
principalmente pelas redes digitais, emergiram as plataformas de
relacionamento online como intermediárias de uma série de interesses,
afetos e desejos das pessoas. A modulação pode ser apresentada como uma
das principais operações que ocorrem nestas plataformas. Modular
comportamentos e opiniões é conduzi"-los conforme os caminhos oferecidos
pelos dispositivos algorítmicos que gerenciam os interesses de
influenciadores e influenciados.

Atualmente, grandes corporações, como o Google, Facebook, Amazon,
Apple, entre outras, concentram as atenções e os fluxos de informação
nas redes digitais. Para vencer a concorrência, coletam permanentemente
dados de seus usuários, traçam seus perfis e tentam mantê"-los fiéis e
atuantes em suas plataformas de interação. Para algumas pesquisadoras e pesquisadores,
somos colocados persistentemente em bolhas com pessoas que
pensam e agem de modo semelhante aos nossos. Para outros analistas, participamos de
diversas amostras que são vendidas para anunciantes que querem conduzir
nossas opções de compra e nosso modo de vida. Maurizio Lazzarato
escreveu no livro \emph{As Revoluções do Capitalismo} que ``a
empresa não cria o objeto (a mercadoria), mas o mundo onde este objeto existe.''

A modulação parece ser uma descrição adequada para um conjunto de
procedimentos realizados nas plataformas digitais. Para modular
opiniões, gostos e incentivar tendências é preciso conhecer muito bem
aquelas pessoas que serão moduladas. Mas, não é possível compreender as
técnicas de modulação com o simplismo das velhas teorias de manipulação.
Modulação é um conceito bem diferente do de manipulação.

Os capítulos desse livro exploram aspectos distintos da complexidade que
pode adquirir o conceito de modulação. No primeiro capítulo, o
pesquisador João Cassino parte da perspectiva do filósofo Gilles Deleuze
para mostrar a constituição do conceito de modulação diferenciando"-o dos
processos de manipulação e indicando o papel dos algoritmos nessa
jornada. No segundo capítulo, o professor Sérgio Amadeu da Silveira vai
descortinar o papel dos algoritmos para conduzir os olhares e a
percepção nas redes de relacionamento \emph{online}. No terceiro
capítulo, a pesquisadora Débora Machado problematiza o poder modulador
das plataformas e suas possibilidades de alterar comportamentos.

Tratando de definir de modo rigoroso a inteligência artificial e o
aprendizado de máquina, a pesquisadora Carla Oliveira mostra, no
capítulo 4, como os algoritmos preditivos estão sendo empregados nesse
cenário de modulação. Já, Cinthia Monteiro, no capítulo 5, produz uma
reflexão sobre a relação entre as práticas de modulação, a biopolítica e
a atual ordem neoliberal que domina o sistema econômico e social. No
capítulo 6, a pesquisadora Mariella Mian busca mostrar o papel da
resistência nas sociedades de controle. Mesmo diante de dinâmicas de
modulação, diversas resistências se afirmam e se multiplicam em um
confronto contínuo.

As reflexões iniciais e exploratórias presentes nessa coletânea são
possíveis porque a universidade pública assegura condições para pesquisa
sem restrições políticas, culturais ou econômicas. Especificamente,
agradecemos e reiteramos a importância da Fapesp que financia uma
pesquisa interdisciplinar sobre a regulação algorítmica no setor
público. Além disso, as bolsas da Capes apoiam pesquisadoras e
pesquisadores que podem se dedicar à atividade científica em nosso País.
Enfim, consideramos que o avanço das ciências depende do
compartilhamento e da colaboração entre aqueles que buscam
compreender a realidade. Por isso, um dos grupos de pesquisa do
Laboratório de Tecnologias Livres (LabLivre), da Universidade Federal do \versal{ABC}
(\versal{UFABC}), apresenta esse esforço inicial de sistematizar o debate sobre um
conceito que acreditamos ser importante para analisar a comunicação e o
processo político nas redes digitais. Esperamos contar com sua leitura
crítica.

\begin{flushright}
\small{
\emph{Joyce Souza}, \emph{Rodolfo Avelino} e \emph{Sérgio Amadeu da Silveira} \emph{(organizadores)}}
\end{flushright}