\chapter{A noção de modulação e os sistemas algorítmicos}

\begin{flushright}
\emph{Sérgio Amadeu da Silveira}
\end{flushright}

Este texto contém uma reflexão sociológica sobre alguns processos
existentes nas plataformas de comunicação e relacionamento online que
podem ser descritos e enquadrados pela noção de modulação. Esse termo
também utilizado no texto de Gilles Deleuze sobre as sociedades de
controle foi resgatado dos escritos sobre tecnologia de Gilbert
Simondon. É importante ressaltar que a direção da reflexão aqui
apresentada não será o debate ontológico, metafísico ou filosófico. A
pretensão, aqui, é levantar alguns pontos para a análise da modulação
como expediente fundamental da comunicação no capitalismo, em sua fase
neoliberal.

Caso sigamos os trajetos das unidades que se comunicam e interagem na
internet logo veremos uma grande concentração de atenções nas
plataformas de relacionamento chamadas de redes sociais online. Somente
o Facebook, a maior dessas plataformas, ultrapassou 2 bilhões de
usuários (2018). No Brasil, em 2017, essas redes sociais foram acessadas
por 78\% das pessoas conectadas à internet, com mais de 10 anos (Cetic,
2018). Plataformas de vídeo online, tal como o Youtube, possuem canais
com mais de 37 milhões de inscritos, sendo extremamente populares em
todos os segmentos sociais.

As plataformas foram adquirindo relevância conforme a internet se
popularizava, principalmente, a partir dos anos 2000, com o sucesso dos
sites que permitiam relacionamentos entre pares, com a explosão do
compartilhamento nas chamadas redes P2P (peer"-to"-peer). O sucesso da
cultura do compartilhamento foi reconhecido pelo mercado que buscou
operar a capitalização desse modelo. A audiência dos sites produtores de
conteúdos foi superada pelas plataformas de interação em que os usuários
produziam as matérias e os objetos. O surgimento e o espraiamento dos
blogs já haviam demonstrado que a colaboração e a produção distribuída
de conteúdos eram práticas envolventes e atraentes de milhares de
pessoas. Em 2003 é lançado o LinkedIn. Em 2004, o Orkut é inaugurado em
janeiro e o Facebook em fevereiro. O Youtube foi criado em 2005 e o
Twitter nasceu em 2006. O êxito dessas plataformas incentivou a
proliferação de modelos de negócios baseados na intermediação entre
ofertantes e demandantes de serviços e mercadorias. O Airbnb surgiu em
2008 e o Uber no ano seguinte.

Em 2009, as redes P2P representavam mais de 50\% do tráfego da internet.
Todavia, a indústria do copyright trocou sua estratégia de
criminalização do compartilhamento de arquivos digitais pela apropriação
privada do trabalho colaborativo, pelo barateamento dos serviços e
produtos digitais e pela negociação das informações dos seus usuários
nos mercados de dados pessoais. Hoje, os serviços pagos em plataformas
de streaming audiovisuais representam mais de 60\% do fluxo do tráfego
nas redes digitais. Em outubro de 2008, o Spotify inicia sua operação.
Em 2011, o Netflix já contava com 23 milhões de assinantes apenas nos
Estados Unidos. O Instagram começa a operar em 2010 e é adquirido pelo
Facebook em 2012. O Waze é lançado em 2008 e adquirido pelo Google em
2013.

A popularização e a queda dos preços dos smartphones e a expansão das
redes wi"-fi ampliaram a conectividade e o tempo médio de utilização da
internet. Esse fato aumentou o poder de intermediação das operadoras de
telecomunicação que havia sido minimizado com a expansão dos serviços de
voz e imagem sobre \versal{IP} da internet, mas não reduziu a força das
plataformas. Em 2016, o faturamento unicamente das quatro corporações
proprietárias das maiores plataformas da internet atingiu \versal{US}\$ 469,3
bilhões (Apple 215,6 bi, Amazon 135,9 bi, Google 90,2 bi e Facebook 27,6
bi). Esse número representa 26\% do \versal{PIB} brasileiro no mesmo período
(\versal{US}\$ 1,796 trilhão)\footnote{Informações obtidas no relatório de 2016
  da Statista: \emph{https://www.statista.com/}.}.

As plataformas ganharam ainda mais poder quanto mais armazenavam dados
dos seus clientes para construírem amostras que permitiam as empresas de
marketing atingirem com precisão aqueles que elas buscavam influenciar.
O microtargenting é muito mais eficiente do que as técnicas massivas de
propaganda. O mundo industrial forjou tecnologias que não eram as mais
propícias para a coleta de dados, mas as tecnologias da informação
permitiam realizar as transações e, simultaneamente, gerar dados sobre
como elas ocorreram e quem as realizou (Nick Srnicek, 2017). O Big Data
e o machine learning e os sistemas algoritmos preditivos aprimoraram a
capacidade de tratar e analisar as informações obtidas nas plataformas.
Eli Pariser (2011) alertou"-nos que essas plataformas filtram nossa
comunicação, analisam nossos comportamentos e nos inserem em bolhas de
pessoas semelhantes.

Um grande mercado de dados e uma microeconomia da interceptação de
informações pessoais se fortaleceu a partir do final da primeira década
do século \versal{XXI} (\versal{SILVEIRA}, 2017). A limitação desse mercado só pode ser
dada pelas legislações de proteção de dados pessoais e pelo direito à
privacidade. Todavia, Shoshana Zuboff (2015) nos mostrou que as
corporações que operam essa economia atuam nos vazios legais e nas
fragilidades das leis e dos seus órgãos de execução. Diante do avanço
das gigantescas plataformas norte"-americanas e chinesas, diversas
empresas dos demais mercados passaram a temer pelos seus negócios e se
somaram aos esforços dos ativistas pelos direitos digitais em busca da
aprovação de legislações de proteção de dados. Mesmo assim, tais peças
legais, em geral, não podem impedir ou simplesmente bloquear as
plataformas que se alimentam de dados pessoais, uma vez que seu tamanho
e a popularização de seu modelo de gratuidade não tem como ser
repentinamente revertidos. Em geral, essas leis baseiam"-se no
consentimento inequívoco e consciente dos usuários de que seus dados
serão coletados e compartilhados. Obviamente, o efeito desse
consentimento é pequeno, pois as pessoas, na maioria das vezes, não têm
opção de negar a entrega de determinados dados diante da necessidade de
uso do serviço.

As plataformas se alimentam de dados pessoais que são tratados e
vendidos em amostras com a finalidade de interferir, organizar o consumo
e as práticas dos seus clientes. Em geral, os conteúdos desses espaços
virtuais são produzidos ou desenvolvidos pelos seus próprios usuários
que, ao mesmo tempo, entregam seus dados pessoais e seus metadados de
navegação para os donos desses serviços. Desse modo, não há nenhum
exagero em nomear o capitalismo informacional em capitalismo de
vigilância (\versal{ZUBOFF}, 2015). Aqui, podemos realçar que a grande
concentração das atenções e do dinheiro dos demais segmentos da economia
nas plataformas se dá porque elas conseguem modular as percepções e os
comportamentos em escala inimaginável até a sua existência.

\section{As plataformas e a modulação do olhar e do afeto}

As teorias funcionalistas da manipulação nasceram nas primeiras décadas
do século \versal{XX} enfatizando que o público poderia ser alvo de mensagens bem
estruturadas que o atingiriam como uma bala mágica levando as massas a
adotarem determinadas opiniões (\versal{DEFLEUR}, 1993). As análises mais
simplistas até as mais sofisticadas, como a hipótese do agendamento ou
\emph{Agenda"-setting theory}\footnote{A \emph{Agenda"-setting theory}, de
  Maxwell McCombs e Donald Shaw, propõe que a cobertura jornalística
  mesmo que não consiga determinar uma opinião, é eficiente para pautar a
  sociedade e fazê"-la pensar sobre um determinado assunto (\versal{MCCOMBS}, 2009).}, trabalham com a produção do discurso, principalmente, das narrativas. A manipulação se dá
fundamentalmente pelo discurso.

As principais plataformas de relacionamento online não produzem
conteúdos. Não realizam discursos, nem criam narrativas. Quem faz o
conteúdo do Facebook, Youtube, Twitter, Instagram, LinkedIn, Snapchat
são seus próprios usuários. Assim, as possíveis tentativas de condução
da opinião e até mesmo de manipulação estariam descentralizadas ou,
ainda, distribuídas nas redes e exclusivamente praticadas pelos usuários
dessas plataformas. Poderíamos até lançar a hipótese de que as
plataformas teriam pouca condição de interferir nos processos de
formação da opinião. Nada mais equivocado.

Aqui pretendo mostrar que a noção de modulação é mais adequada para
tratar dos processos de formação de opinião nas plataformas de
relacionamento online, especialmente, nas chamadas redes sociais. No
mundo pré"-internet, o discurso das mídias era o que adquiria maior
impacto. A escassez induzida de canais de comunicação corroborava com a
concentração das atenções em um conjunto de produtores e distribuidores
de narrativas. No mundo da internet, na fase do predomínio das
plataformas, os conteúdos são produzidos de modo distribuído, mesmo que
assimétricos, e por elas organizados.

A organização daquilo que é postado e disposto nos circuitos fechados
das plataformas não é realizado livremente pelos seus criadores. As
plataformas possuem sua própria arquitetura de informação que é
centralizada, completamente diferente da topologia distribuída da
internet. O fluxo de acesso aos conteúdos também é definido pelos
gestores das plataformas. A descrição do sociólogo Manuel Castells
parece descortinar o processo de controle existente nessas redes
fechadas:

\begin{quote}
Em um mundo de redes, a capacidade para exercer controle sobre os outros
depende de dois mecanismos básicos: 1) a capacidade de constituir redes
e de programar/reprogramar as redes segundo os objetivos que lhes
atribuam; e 2) a capacidade para conectar diferentes redes e assegurar
sua cooperação compartilhando objetivos e combinando recursos, enquanto
se evita a competência de outras redes estabelecendo uma cooperação
estratégicas (\versal{CASTELLS}, 2009, 76).
\end{quote}

As plataformas reúnem pessoas que querem ou necessitam se agrupar ou
pertencer a redes de amizade, negócios, afetos, entretenimento. Como
integrantes, essas pessoas tem o poder de entrar ou abandonar a
plataforma, muito diferente do poder que os gestores ou que os donos
desses redes privadas detém. Um dos principais modos de controle que os
gestores das plataformas possuem sobre seus usuários se dá pela
modulação das opções e dos caminhos de interação e de acesso aos
conteúdos publicados.

A modulação é um processo de controle da visualização de conteúdos,
sejam discursos, imagens ou sons. As plataformas não criam discursos,
mas possuem sistemas algoritmos que distribuem os discursos criados
pelos seus usuários, sejam corporações, sejam pessoas. Assim, os
discursos são controlados e vistos, principalmente, por e para quem está
dentro dos critérios que constituem as políticas de interação desses
espaços virtuais. Para engendrar o processo de modulação não é preciso
criar um discurso, nem uma imagem ou uma fala, apenas é necessário
encontrá"-los e destiná"-los a segmentos da rede ou a grupos específicos,
conforme critérios de impacto e objetivos previamente definidos.

Para modular é necessário reduzir o campo de visão dos indivíduos ou
segmentos que serão modulados. É preciso oferecer algumas alternativas
para se ver. A modulação encurta a realidade e a multiplicidade de
discursos e serve assim ao marketing. Os sistemas algoritmos filtram e
classificam as palavras"-chaves das mensagens, detectam sentimentos,
buscam afetar decisivamente os perfis e, por isso, organizam a
visualização nos seus espaços para que seus usuários se sintam bem,
confortáveis e acessíveis aos anúncios que buscarão estimulá"-los a
adquirirem um produto ou um serviço. A modulação opera pelo encurtamento
do mundo e pela oferta, em geral, de mais de um caminho, exceto se ela
serve aos interesses de uma agência de publicidade, instituição ou uma
corporação compradora. Assim, ficamos quase sempre em bolhas que prefiro
chamar de amostras, filtradas e organizadas conforme os compradores, ou
melhor, anunciantes.

Para que o processo de modulação seja eficiente e eficaz, as plataformas
precisam conhecer bem cada um que interage em seus espaços ou
dispositivos. Por isso, a modulação é um recurso"-procedimento do mercado
de dados pessoais e um estágio na cadeia da microeconomia da
interceptação de dados pessoais. A captura ou a colheita de dados é o
primeiro passo. O armazenamento e a classificação desses dados devem ser
seguidos pela análise e formação de perfis. Diversos bancos de dados
podem ser agregados a um perfil pelas possibilidades trazidas pelo Big
Data. Os sistemas algorítmicos modelados como aprendizado de máquina
devem acompanhar os clientes das plataformas em cada passo, reunindo
informações precisas sobre os cliques dados, os links acessados, o tempo
gasto em cada página aberta, os comentários apagados, entre outros.

O processo de modulação começa por identificar e conhecer precisamente o
agente modulável. O segundo passo é a formação do seu perfil e o
terceiro é construir dispositivos e processos de acompanhamento
cotidiano constantes, se possível, pervasivos. O quarto passo é atuar
sobre o agente para conduzir o seu comportamento ou opinião. Para
ilustrar esse processo, vamos observar a patente da Samsung denominada
\emph{Apparatus and method for determining user's mental state}, em
português, ``Aparelho e método para determinar o estado mental do
usuário''. A solicitação de patente registrada no escritório coreano em
9 de novembro de 2012 e no escritório norte"-americano, em 8 de novembro
de 2013, com o número US9928462B2, permite obter informações
fundamentais para o processo de modulação, seja na formação do perfil,
seja no acompanhamento cotidiano do agente. Seu resumo é esclarecedor:

\begin{quote}
Um aparelho para determinar o estado mental de um usuário em um terminal
é fornecido. O aparelho inclui um coletor de dados configurado para
coletar dados do sensor; um processador de dados configurado para
extrair dados de recursos do sensor; e um determinador de estado mental
configurado para fornecer os dados do recurso a um modelo de inferência
para determinar o estado mental do usuário\footnote{\emph{An apparatus for
  determining a user's mental state in a terminal is provided. The
  apparatus includes a data collector configured to collect sensor data;
  a data processor configured to extract feature data from the sensor
  data; and a mental state determiner configured to provide the feature
  data to an inference model to determine the user's mental state}
  (US9928462B2, tradução livre).} (US9928462B2).
\end{quote}

A patente é um instituto importante do capitalismo. Corporações
registram patentes para aplicá"-las e também para impedir que os
concorrentes utilizem aqueles modelos, inventos e procedimentos, exceto
se remunerem o seu titular, aquele que a registrou e passa a ser seu
dono. Todavia, nem todas as patentes são utilizadas, muitas servem para
bloquear uma tecnologia ainda em pesquisa ou simplesmente para ser
transformadas em munição em uma guerra contra outras empresas. Como os
procedimentos e os sistemas algoritmos das plataformas são obscuros, a
análise do texto das patentes, mesmo genérico, podem nos ajudar a
compreender a dinâmica ofuscada e invisível aos usuários.

A descrição da patente em questão nos permite compreender o potencial de
modulação dos dispositivos mediadores de nossa comunicação. Está escrito
que:

\begin{quote}
(\ldots{}) o estado mental pode incluir uma ou mais de uma emoção, um
sentimento e um estresse, cada um dos quais pode ser classificado em
vários níveis inferiores. Por exemplo, emoção pode ser classificada em
felicidade, prazer, tristeza, medo, etc.; sentimento pode ser
classificado em bom, normal, deprimente, etc.; e o estresse pode ser
classificado em alto, médio e baixo\footnote{Tradução livre da
  descrição: \emph{the mental state may include one or more of an emotion, a
    feeling, and a stress, each of which may be classified into various
    lower levels. For example, emotion may be classified into happiness,
    pleasure, sorrow, fright, etc.; feeling may be classified into good,
    normal, depressing, etc.; and the stress may be classified into high,
    medium, and low}.}.
\end{quote}

Mas como é possível identificar tais sensações e sentimentos? A patente
nos dá uma indicação:

\begin{quote}
(\ldots{}) quando a velocidade de digitação usando um teclado é de 23
caracteres por minuto, a freqüência de uso da tecla de retrocesso é três
vezes ao escrever uma mensagem, a freqüência de uso de um sinal especial
é cinco vezes, o número de tremores de um dispositivo é 10, uma
iluminância média é de 150 Lux, e um valor numérico de uma localização
específica (por exemplo, estrada) é 3, um estado de emoção classificado
aplicando os dados do recurso ao modelo de inferência é ``susto'', com
um nível de confiança de 74\%\footnote{Tradução livre da descrição:
  \emph{``\ldots{} when typing speed using a keyboard is 23 characters per
    minute, the frequency of use of the backspace key is three times while
    writing a message, the frequency of use of a special sign is five
    times, the number of shakings of a device is 10, an average
    illuminance is 150 Lux, and a numerical value of a specific location
    (for example, road) is 3, an emotion state classified by applying the
    feature data to the inference model is ``fright,'' with a confidence
    level of 74\%''.}} (US9928462B2).
\end{quote}

O conhecimento do estado emocional dos agentes é um dos elementos
importantes para que o processo de modulação seja bem"-sucedido. Existem
5.162 patentes consideradas similares a patente da Samsung, aqui
descrita, registras ou aguardando o registro final nos principais
escritórios de patentes\footnote{Essa informação foi obtida do buscado
  do Google disponível para as buscas de patentes.}. Destas patentes
similares, 7,4\% são da Samsung Electronics Coreana; 4,5\% são da
Samsung Electronics dos \versal{EUA}; 3,9\% são da Microsoft Technology
Licensing, Llc; 3,3\% do Google Inc.; 3,2\% da Microsoft Corporation;
3,1\% da Apple Inc.; 2,7\% do Facebook, Inc.; 2,5\% da \versal{IBM}; 1,1 \% da
Amazon Technologies, Inc.; 1\% do Linkedin Corporation; 1\% do Ebay
Inc.; 1\% do Yahoo! Inc.

As plataformas online possuem outras patentes esclarecedoras e que
corroboram com a definição do processo de modulação aqui descrito. São
milhares delas, aqui apresento mais cinco, cuja denominação é suficiente
para termos uma noção de sua finalidade:

\begin{itemize}
\item
  US-2010088607-A1 -- Sistema e método para manter o usuário sensível ao
  contexto (Yahoo);
\item
  US-2012272338-A1 -- Gerenciamento unificado de dados de rastreamento
  (Apple);
\item
  US-2012226748-A1 -- Identifique Especialistas e Influenciadores em uma
  Rede Social (Facebook);
\item
  US2018019226-3A1 -- Prever o estado futuro de um usuário de dispositivo
  móvel, (Facebook);
\item
  US20080033826-A1 -- Fornecimento de anúncios baseados no humor e na
  personalidade (Pudding Ltd);
\end{itemize}

Com a utilização de algoritmos, principalmente de \emph{machine
learning}, as plataformas conseguem estruturar processos de modulação
que são desenvolvidos para delimitar, influenciar, reconfigurar o
comportamento dos interagentes na direção que os mantenha disponíveis e
ativos na plataforma ou que os faça clicar e adquirir os serviços,
produtos e ideias negociadas pelos donos do empreendimento. A modulação
depende dos sistemas algoritmos e da estrutura de dados ampla, vasta e
variada dos viventes, dentro e fora das plataformas e redes digitais.
Como nos alertou Deleuze (1992), a modulação passa a ser continua e o
marketing se tornou a principal forma de controle social.

\section{Neoliberalismo e modulação}

O neoliberalismo é a atual doutrina do Capital. Pode ser visto como a
nova racionalidade do capitalismo (\versal{DARDOT} e \versal{LAVAL}, 2017). A doutrina
neoliberal interfere e tem implicações no desenvolvimento da internet e
de suas invenções. Além disso, o pensamento neoliberal opera nas redes
digitais e plataformas com a finalidade de anular e dissipar todas as
ações coletivas que criem outras lógicas que não sejam voltadas à
concorrência e a reprodução do Capital. Os processos de
espetacularização que já existiam no mundo industrial se intensificaram
no cenário informacional e foram reforçados nas redes sociais embaladas
pelo contexto neoliberal.

A modulação nas plataformas digitais tem servido, principalmente, à
expansão do neoliberalismo. O marketing utiliza as corporações para
moldar nossas subjetividades e formatar nossos afetos. Robôs tem lido
nossos e"-mails mais íntimos e apresentado respostas possíveis ao nosso
remetente. Isso passa desapercebido para grande parte das pessoas e tem
sido compreendido como ``algo natural da tecnologia''. O poder de
tratamento das informações é legitimado por um entorpecimento subjetivo
diante das vantagens oferecidas pelas tecnologias apresentadas pelas
corporações. São tecnologias que reforçam o que Guattari chamou de
servidão maquínica. Ao organizar nossas práticas cotidianas em torno
dessas corporações passamos de utilizadores à dependentes de suas
tecnologias.

A lógica da concorrência foi apontada por Foucault (2008) como a lógica
estruturante do pensamento neoliberal. As pessoas permanentemente
conectadas têm seus dados sucessivamente coletados por sistemas
algorítmicos que culminarão em processos de modulação extremamente úteis
a aceleração da concorrência. Quem não conhecer profundamente seus
possíveis consumidores será derrotado ou engolido no cenário neoliberal,
por isso a crescente aposta nessa microeconomia da intrusão e da
interceptação de dados pessoais.

As técnicas de modulação são imprescindíveis para as corporações
praticarem o marketing certeiros, específico e personalizado. Quanto
mais dependente dos dispositivos tecnológicos que coletam dados, mais as
pessoas terão seus perfis comportamentais e opinativos organizados e
analisados como parte de um processo que culminará no encurtamento do
mundo, da condução da visão e na entrega de opções delimitadas. Os
sistemas algorítmicos preditivos das plataformas querem conhecer cada
vez mais as pessoas para melhor atendê"-las e ``fidelizá"-las''. A munição
dessa guerra concorrencial são os dados obtidos de cada pessoa para
nutrir o processo de modulação, sem o qual não será possível se manter,
nem vencer os concorrentes.

O neoliberalismo se fortalece na modulação, mas também gera
resistências. Suas tentativas de redução da vida à concorrência de
mercado e ao enaltecimento da empresa como unidade principal e basilar
da sociedade é denunciado na esfera pública automatizada (\versal{PASQUELE},
2017). Nesse sentido, nas redes digitais e nas plataformas as modulações
do Capital e de suas forças político"-culturais convivem com
tecnopolíticas anti"-sistêmicas e com articulações pós"-capitalistas que
ainda não conseguiram superar o axioma do Capital, mas resistem a sua
supremacia.

\section{Referências}

\versal{CASTELLS}, Manuel. \emph{Power communication.} New York: Ed. Oxford,
2009.

\versal{DEFLEUR}, Melvin. \emph{Teorias da comunicação de massa}. Zahar, 1993.

\versal{DELEUZE}, Gilles. \emph{Sobre as Sociedades de Controle Post-Scriptum}. \emph{In}:
\versal{DELEUZE}, Gilles. \emph{Conversações}. Trad. Peter Pál Pelbart. São Paulo:
Editora 34, p. 219-226, 1992.

\versal{DARDOT}, Pierre; \versal{LAVAL}, Christian. \emph{A nova razão do mundo}.
Boitempo Editorial, 2017.

\versal{FOUCAULT}, Michel. \emph{Nascimento da Biopolítica: curso dado no
Collège de France (1977-1978)}. São Paulo: Martins Fontes, 2008.

\versal{MCCOMBS}, Maxwell. \emph{A teoria da agenda: a mídia e a opinião
pública}. Petrópolis: Vozes, 2009.

\versal{PARISER}, Eli. \emph{The filter bubble: What the Internet is hiding
from you}. Londres: Penguin \versal{UK}, 2011.

\versal{SILVEIRA}, Sérgio Amadeu da. \emph{Tudo sobre tod@ s: Redes digitais,
privacidade e venda de dados pessoais}. Edições Sesc, 2017.

\versal{SIMONDON}, Gilbert. \emph{On the mode of existence of technical
objects}. Deleuze Studies, v. 5, n. 3, p. 407-424, 2011.

\versal{SRNICEK}, Nick. \emph{Platform capitalism}. John Wiley \& Sons, 2017.

\versal{UGARTE}, David de. \emph{O poder das redes: manual ilustrado para
pessoas, organizações e empresas, chamadas a praticar o ciberativismo}.
Porto Alegre: \versal{EDIPUCRS}, 2008.

\versal{ZUBOFF}, Shoshana. \emph{Big other: surveillance capitalism and the
prospects of an information civilization}. Journal of Information
Technology, v. 30, n. 1, p. 75-89, 2015.\\

\versal{PATENTES CITADAS}

US9928462B2 -- Apparatus and method for determining user's mental state.

US-2010088607-A1 -- System and method for maintaining context sensitive
user

US-2012272338-A1 -- Unified tracking data management

US-2012226748-A1 -- Identify Experts and Influencers in a Social Network

US2018019226-3A1 -- Predicting the future state of a mobile device user

US20080033826-A1 -- Personality-based and mood-base provisioning of
advertisements
